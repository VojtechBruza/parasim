\documentclass{article}
\usepackage{template}

\begin{document}
\header

\section{Zpráva za období duben až červen 2012}

\subsection{První etapa}

V první etapě našeho projektu jsme se zaměřili na integraci, dokončení modulů,
z nichž se nástroj skládá, a tvorbu jednoduchého rozhraní
z příkazové řádky. Většinu vytyčených úkolů jsme stihli do května 2012, avšak
kvůli vytížení přes zkouškové období byla první verze vydána až o měsíc později.
Ve zkratce lze říci, že bylo vyřešeno následující:

\begin{itemize}
    \item   načítání vstupů a potřebné úpravy modelu,
    \item   export a vyzualizace výsledků,
    \item   dokončení modulů pro výpočet robustnosti, zahušťování a simulaci,
    \item   potřebné úpravy ve frameworku,
    \item   vytvoření obecného modelu pro výpočet a implementace algoritmu pro analýzu dynamických systémů.
\end{itemize}


Vydaná verze je k dispozici na stránkách našeho projektu\footnote{\url{https://github.com/sybila/parasim/zipball/1.0.0.M1}},
kde najdete též seznam vyřešených úkolů\footnote{\url{https://github.com/sybila/parasim/issues?milestone=1&state=closed}}.

\subsection{Druhá etapa}

Ve druhé etapě, která končí v červenci 2012, se zabýváme tvorbou testů a dokumentace,
hledáním vhodných modelů pro analýzu, profilováním a drobnými úpravami nástroje.

Kompletní seznam úkolů pro tutu etapu je k dispozici na stránkách projektu\footnote{\url{https://github.com/sybila/parasim/issues?milestone=4&page=1}}.

\end{document}
